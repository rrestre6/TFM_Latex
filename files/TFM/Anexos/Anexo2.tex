
\tiny
\begin{lstlisting}[language=Matlab, caption={Rutina para simular la propagaci�n de la luz.}, captionpos=b, label=Propanexo2]
%------------------------------------------------------------------------------
%% atmos_setup

% determine geometry
D2 = 0.235;   % diameter of the observation aperture [m]
wvl = 0.55e-6; %1e-6; % optical wavelength [m]
k = 2*pi / wvl; % optical wavenumber [rad/m]
Dz =  20e3; %50e3;    % propagation distance [m]

% use sinc to model pt source
DROI = 4 * D2;  % diam of obs-plane region of interest [m]
D1 = wvl*Dz / DROI;    % width of central lobe [m]
R = Dz; % wavefront radius of curvature [m]

% atmospheric properties
Cn2 = 1e-15; It is a average over air column % structure parameter [m^-2/3]
% SW and PW coherence diameters [m]
r0sw = (0.423 * k^2 * Cn2 * 3/8 * Dz)^(-3/5);
r0pw = (0.423 * k^2 * Cn2 * Dz)^(-3/5);
p = linspace(0, Dz, 1e3);
% log-amplitude variance
rytov = 0.563 * k^(7/6) * sum(Cn2 * (1-p/Dz).^(5/6) ...
    .* p.^(5/6) * (p(2)-p(1)));

% screen properties
nscr = 8; % number of screens
A = zeros(2, nscr); % matrix
alpha = (0:nscr-1) / (nscr-1);
A(1,:) = alpha.^(5/3);
A(2,:) = (1 - alpha).^(5/6) .* alpha.^(5/6);
b = [r0sw.^(-5/3); rytov/1.33*(k/Dz)^(5/6)];
% initial guess
x0 = (nscr/3*r0sw * ones(nscr, 1)).^(-5/3);
% objective function
fun = @(X) sum((A*X(:) - b).^2);
% constraints
x1 = zeros(nscr, 1);
rmax =  1; % maximum Rytov number per partial prop. 
x2 = rmax/1.33*(k/Dz)^(5/6) ./ A(2,:);
x2(A(2,:)==0) = 50^(-5/3);
% options = optimoptions('fmincon','Display','iter','Algorithm','sqp');
options = optimset('Display','iter');
[X,fval,exitflag,output] ...
    = fmincon(fun,x0,[],[],[],[],x1,x2,[],options);
% check screen r0s
r0scrn = X.^(-3/5);
r0scrn(isinf(r0scrn)) = 1e6;
% check resulting r0sw & rytov
bp = A*X(:); [bp(1)^(-3/5) bp(2)*1.33*(Dz/k)^(5/6)]
[r0sw rytov]

%------------------------------------------------------------------------------
% pt_source_vac_prop

delta1 = d1;    % source-plane grid spacing [m]
deltan = d2;    % observation-plane grid spacing [m]
n = nscr;         % number of planes

% coordinates
[x1 y1] = meshgrid((-N/2 : N/2-1) * delta1);
[theta1 r1] = cart2pol(x1, y1);

% point source
pt = exp(-i*k/(2*R) * r1.^2) / D1^2 ...
   .* sinc(x1/D1) .* sinc(y1/D1) ...
   .* exp(-(r1/(4*D1)).^2);

%pt=IxScl;
% partial prop planes
z = (1 : n-1) * Dz / (n-1);

% simulate vacuum propagation
sg = exp(-(x1/(0.47*N*d1)).^16) ...
    .* exp(-(y1/(0.47*N*d1)).^16);
t = repmat(sg, [1 1 n]);
[xn yn Uvac] = ang_spec_multi_prop(pt, wvl, ...
    delta1, deltan, z, t);
% collimate the beam
Uvac = Uvac .* exp(-i*pi/(wvl*R)*(xn.^2+yn.^2));

%------------------------------------------------------------------------------
%% pt_source_turb_prop

l0 = 0;     % inner scale [m]
L0 = inf;   % outer scale [m]

zt = [0 z];  % propagation plane locations
Delta_z = zt(2:n) - zt(1:n-1);    % propagation distances
% grid spacings
alpha = zt / zt(n);
delta = (1-alpha) * delta1 + alpha * deltan;

% initialize array for phase screens
phz = zeros(N, N, n);
nreals = 10;    % number of random realizations
% initialize arrays for propagated fields,
% aperture mask, and MCF
Uout = zeros(N);
mask = circ(xn/D2, yn/D2, 1);
MCF2 = zeros(N);
sg = repmat(sg, [1 1 n]);
for idxreal = 1 : nreals     % loop over realizations
    idxreal
    % loop over screens
    for idxscr = 1 : 1 : n
        [phz_lo phz_hi] ...
            = ft_sh_phase_screen ...
            (r0scrn(idxscr), N, delta(idxscr), L0, l0);
        phz(:,:,idxscr) = phz_lo + phz_hi;
    end
    % simulate turbulent propagation
    [xn yn Uout] = ang_spec_multi_prop(pt, wvl, ....
        delta1, deltan, z, sg.*exp(i*phz));
    % collimate the beam
    Uout = Uout .* exp(-i*pi/(wvl*R)*(xn.^2+yn.^2));
    % accumulate realizations of the MCF  % Mirar Statistical Optics MCF=OTF
    MCF2 = MCF2 + corr2_ft(Uout, Uout, mask, deltan);
end
% modulus of the complex degree of coherence
MCDOC2 = abs(MCF2) / (MCF2(N/2+1,N/2+1));

%------------------------------------------------------------------------------
%% Telescope Imaging (PSF) and Atsmosphere imaging

OTF_Telsecope = corr2_ft(mask, mask, ones(N), deltan);
PSF_Telescope = ift2(OTF_Telsecope, deltan);
PSF_Telescope = PSF_Telescope/max(PSF_Telescope(:));

PSF_Atmosphere = ift2(MCF2, deltan);
PSF_Atmosphere = PSF_Atmosphere/max(PSF_Atmosphere(:));

%OTF = ((OTF_Telsecope/max(OTF_Telsecope(:))).*(MCF2/max(MCF2(:)))).*mask;
%PSF = ift2(OTF, deltan);
%PSF = PSF/max(PSF(:));

%------------------------------------------------------------------------------
%% Units convertion

% Telescope properies Aperture D2 = 235 mm. conFactor paralax deducted (pc/AU). 
% Pixel size in microns
fRatio=10; focal=2350; covFactor=206265; pixelSize=4.8; 
S=covFactor/focal; % Angular magnification
rs=S*10^-3*pixelSize; % Resolution
th=1.22*covFactor*(wvl/D2); % Angular resolution
PSFcover= ceil(th/rs);
PSF_Tel_1D = real(PSF_Telescope(:,size(PSF_Telescope,1)/2+1));
PSF_size=PSF_Tel_1D( PSF_Tel_1D(:) >= 0.02 & PSF_Tel_1D(:) < 1);
realUnits_PSF = PSFcover/length(PSF_size);
scaledPSF_Atmosphere = imresize(real(PSF_Atmosphere) , realUnits_PSF, 'nearest');
if mod(size(scaledPSF_Atmosphere,1),2) == 1;
    scaledPSF_Atmosphere =  padarray(scaledPSF_Atmosphere,[1 1],0,'pre');
end
h= fspecial('average',7);
scaledPSF_Atmosphere = imfilter(scaledPSF_Atmosphere,h,'same'); 
scaledPSF_Atmosphere =  padarray(scaledPSF_Atmosphere,[(size(PSF_Telescope,1)/2 ...
	-round(size(scaledPSF_Atmosphere,1)/2)) ...
    (size(PSF_Telescope,1)/2-round(size(scaledPSF_Atmosphere,1)/2))],0,'both');
    
\end{lstlisting}
